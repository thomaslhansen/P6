% !TeX root = ../main.tex

% This chapter is aimed to introduce both the subject of the project as well as our motivation for the project.
% It also aims to establish the initial problem statement, which the analysis will be conducted based on.
% This section should describe the basis of our project.
% really where our solution originates, based on a real world problem,
% even though our project is based on a "fun" project, it should still be possible to abstract it into real world applications
\chapter{Introduction}

This project is on a very different form than anything we have previously worked on during our studies.
The introduction has therefore been split into different sections, each describing the different aspects of the project.
First a short section about the project itself and its motivation followed by a section about how the process work should be conducted and ending with a section about the start up process of the project period.

\section{About the Giraf project}
The Giraf (Graphical Interface Resource for Autistic Folk) project is a software solution for autistic folk with little to no verbal communication.
The software solution is primarily used as a scheduling assistant, but the solution also aims to assist in education and games.
The project has been developed by 6\textsuperscript{th} semester Software engineering students since 2011 and has changed a lot over the years since its initiation.
When a project period is completed, the project is passed along to the following year's 6\textsuperscript{th} semester software engineering students.

The semester coordinator Ulrik Mathias Nyman has had the overview of the project since its beginning in 2011.
Nyman expressed a lot of engagement in the start-up of the project in order to facilitate a smooth transition from the usual type of semester projects that the students have been working in, to this new multi-group project work.
Nyman also facilitated meetings and contact with the faculty at AAU to make the work for the students as effortless as possible.

The Giraf-project is an obvious learning opportunity for the students to experience how working with multiple costumers will result in unforeseen problems along the way.

Throughout the project there have been multiple costumers:

\begin{itemize}
	\item Børnehaven Birken (a local special needs kindergarten)
	\item Egebakken (a local special needs school)
	\item Fagcenter for Autisme og ADHD (a part of the municipality that facilitates activities for people with ADHD or autism)
	\item IT-chef i Ældre- og Handicapforvaltningen (the IT department manager in the Elderly- and Disability-management)
\end{itemize}

These costumers play a large role during the entire project, since it is their needs/wishes that the project backlog will be based upon.

\section{Scrum of Scrums}
Since this project is a multi group, semester wide project, a team was made responsible for managing the work process.
This group decided that the entire project should be run as a scrum project, but that the individual groups were to decide how they would work.

While one group was responsible for the work process, another group was responsible for facilitating the contact with the costumers, hence becoming the product owners of the project.
They were also responsible for planning each sprint so that the costumers would get the most amount of value as fast as possible.
All groups were also participating in the planning of each sprint, as only they knew how much time they had for project work and an estimate of how many story points they could complete during the sprint.

During a sprint each group were to send one representative to a weekly scrum stand up so that all the groups knew what each group was doing.
After each sprint a sprint retrospective and review meeting were held in order for the groups to share their experiences from the last sprint for the next sprint period.
The outcome of this meeting were the points to make better in the next sprint period.

In our group we would also follow the scrum guidelines in order to produce a better product with a higher degree of agility in anticipation of changing requirements from the costumers.
Hence the name Scrum of Scrums, since we had multiple scrums processes running independent of each other.
\section{Semester roles}
Since the project is so large with regards to the number of participants, the group responsible for the work process decided that all teams were to be full stack teams.
The advantage of this is that each group contains the capabilities to complete all aspects of an issue from the project backlog.
Based on the old reports from the last few semesters who had been working on this project, the main problem was that one group would be waiting on one or more other groups to finish their work, before they could begin on their own task.
Another problem earlier semesters faced was that some groups did not feel as if they were a part of the project, since they would only be working on their own little niche part of the project without interacting with the rest of the semester groups.

This new approach lead to the responsibilities of each group to be divided, as can be seen on Figure~\ref{TBL:GroupResponsibility}.
\begin{table}[H]
\centering
\begin{tabular}{|l|l|}
\hline
\textbf{Group} & \textbf{Responsibility} \\ \hline
SW602F19       & Process                 \\ \hline
SW608F19       & Full stack              \\ \hline
SW609F19       & Full stack              \\ \hline
SW610F19       & Product Owner           \\ \hline
SW611F19       & Full stack              \\ \hline
SW612F19       & Full stack              \\ \hline
SW613F19       & Full stack              \\ \hline
\end{tabular}
\captionof{table}{Table showing the different responsibilities across the groups}\label{TBL:GroupResponsibility}
\end{table}

\section{Meta groups}
To facilitate these full-stack groups, the addition of meta groups was made.
The main reason the meta-groups were formed was that in the previous semesters some groups felt left out.
These groups will have the function of facilitating the full-stack work so the groups have a better connection, cooperation and a common understanding of the current progress of the project.
There were three meta groups introduced in the semester:

The \textit{frontend} meta group is responsible for everything regarding the user interface, they will be taking decisions for all of the groups in regards to UI design.

The \textit{backend} meta group will be the responsible party when it comes to all the functionality that is hidden away behind the UI, and the decisions that go along with it.

The \textit{server} meta group take the required decisions about server infrastructure, and manages the aquisition of servers from AAU.

A group member was assigned to each of the meta groups in order to keep track of what was going on in each of the groups. Meetings will be held to keep each group informed about what is currently happening in the project, across all of the groups. These meetings will also be used to make concrete decisions in regard to changes affecting the entire Giraf project.
\section{Tools}
As a part of this project, a number of different tools were used to facilitate the development done by the different groups.
In the following each of the different tools will be described. 

\subsection{GitHub}
For hosting the code of the Giraf project, the previous semesters has been using GitLab in a self hosted environment. 
It was chosen to migrate from the self hosted GitLab to the regular GitHub.
This was primarily done such that no one working on the project have to maintain the GitLab instance and servers.
An organisation was setup on GitHub, and all the different members of the project would then add a team to the organization. On \autoref{GitHubOrg} the front page of the organization can been seen. 

\figur{1}{images/GitHub.png}{GitHub Giraf organization main page}{GitHubOrg}

Each developer group were added as GitHub Teams, as can be seen on \autoref{GitHubTeams}, which made it easier to assign the different groups to a given issue, or simply to tag the entire group in a comment. 

\figur{1}{images/GitHubTeams.png}{GitHub Giraf team overview}{GitHubTeams}

\subsection{Azure Pipeline}
With GitHub it is possible to integrate different solutions for testing whether or not a given branch is up to the quality standard. 
Before each branch is able to merge it needs to pass two criteria. 
First it needs to be reviewed by two developers and one representative from the product owner group. 
Additionally, all checks implemented in the Azure pipeline would need to pass, as can be seen on \autoref{GitHubMergeControl}. 

\figur{1}{images/GitHubMergeControl.png}{GitHub merge control}{GitHubMergeControl}

The checks implemented were first of all the Flutter linter, which would check that the code design followed Flutter guidelines. 
The pipeline would also check if the flutter-compiler could compile without any warnings/errors. 
On \autoref{GitHubPipeline} the build result of a pipeline run can be seen.

\figur{1}{images/GitHubPipeline.png}{GitHub Azure Pipeline integration}{GitHubPipeline}


\subsection{Docker}
The group was largely in charge of making the servers work with a minimal down time, i.e. time where the services are unavailable.
For this purpose, Docker Community Edition was chosen as the solution for hosting the different services of the project. 
Docker is a container platform that is able to mange large applications with ease. 
For a production setup, Docker Swarm is recommended as it is able to set up the services (and manage them once running), able to handle service cashes, and even able to do ongoing health checks to evaluate the health of the service. 

Once the Docker Swarm has been created, the developers are able to virtualize the same production environment on each of their own local machines. 

Docker uses an online repository to mange Docker Images from which it spins up the services. 
For this project, a Docker Hub organization was created to share the custom images written for the project. 
An overview of the organization can be seen on \autoref{DockerHub}

\figur{1}{images/DockerHub.png}{Docker Hub Organization}{DockerHub}

\subsection{Slack}
For internal communication - both in the project group it self but also across all the other groups - the Slack chat service was used as the primary means of communication. 
Slack offers the ability to create channels for each group as well as channels for more general purpose communication. 
On \autoref{Slack} a selection of the channels can be seen. 

\figur{0.3}{images/Slack.png}{Slack channels}{Slack}

Slack does have the ability to integrate with other services such as GitHub and Google Calendar, this was, however, not used since it was deemed as an unnecessary noise, which would make it harder for more important messages to get any attention. 

\subsection{Email}
For more formal communication outside the project itself, email was used as the official communication channel. 
This was primarily used then communicating with other AAU employees and staff. 
When using email as the mean of communication it was of most importance to use the CC function so every group member would also receive the emails.


\section{Project start up}
Before the first sprint could start the two groups responsible for both process and costumer contact were given time to both formalise the common work process and to contact the costumers.
Once both groups where ready they were to present their work to the rest of the groups.

Once all groups were aware of the work process and the backlog of the project the first sprint could be planned and stated as described in Part~\ref{PART:Sprints}.
