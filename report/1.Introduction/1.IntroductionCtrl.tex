% !TeX root = ../main.tex
% This chapter is aimed to introduce both the subject of the project as well as our motivation for the project.
% It also aims to establish the initial problem statement, which the analysis will be conducted based on.
% This section should describe the basis of our project.
% really where our solution originates, based on a real world problem,
% even though our project is based on a "fun" project, it should still be possible to abstract it into real world applications
\chapter{Introduction}
This project is on a very different form than anything we have previously worked on during our studies.
The introduction has therefore been split into different sections, each describing the different aspects of the project.
First a short section about the project itself and its motivation.
Later a section about the start up process of the project work.

\section{About the Giraf project}
The Giraf (Graphical Interface Resource for Autistic Folk) project is a software solution for autistic folk with little to no verbal communication.
The software solution is primarily used as a scheduling assistant, but the solution also aims to assist in education and games.
The project has been developed by 6\textsuperscript{th} semester Software engineering students since 2011 and has changed a lot over the years.
After each project period the project is passed along to the next 6\textsuperscript{th} semester software engineering students.

Since the project has been running since 2011, the semester coordinator Ulrik Mathias Nyman  has had the overview of the project since the beginning.
He has been very engaged in the start up of the project at each semester in order to facilitate a smooth transition from the "normal" semester projects that the students has been used to, to this new multi group project work.
He has throughout the project been facilitating meetings and contact with the faculty to make the work for the students as effortless as possible.

The project is an excellent learning opportunity for the students to learn that working with multiple costumers will result in unforeseen problems along the way.

Throughout the project there has been multiple costumers:

\begin{itemize}
	\item Børnehaven Birken, the local kindergarten
	\item Egebakken, the local school
	\item Fagcenter for Autisme og ADHD, a local center that facilitates activities for people with ADHD or autism
	\item IT-chef i Ældre- og Handicapforvaltningen, the IT chief in the Elderly- and Disability-management
\end{itemize}

These costumers will play a large role during the project period since it is their needs that will be inserted into the project backlog.
Hence all work that is done on the project will be done to facilitate their needs/wishes.

\section{Scrum of Scrums}
Since this project is a multi group, semester wide project, a team was made responsible for managing the work process.
This group decided that the entire project should be run as a scrum project, but that the individual groups were to decide how they would work.

While one group was responsible for the work process, another group was responsible for facilitating the contact with the costumers, hence becoming the product owners of the project.
They were also responsible for planning each sprint so that the costumers would get the most amount of value as fast as possible.
All groups were also participating in the planning of each sprint, as only they knew how much time they had for project work and an estimate of how many story points they could complete during the sprint.

During a sprint each group were to send one representative to a weekly scrum stand up so that all the groups knew what each group was doing.
After each sprint a sprint retrospective and review meeting were held in order for the groups to share their experiences from the last sprint for the next sprint period.
The outcome of this meeting were the points to make better in the next sprint period.

In our group we would also follow the scrum guidelines in order to produce a better product with a higher degree of agility in anticipation of changing requirements from the costumers.
Hence the name Scrum of Scrums, since we had multiple scrums processes running independent of each other.

\section{Semester roles}
Since the project was so large with regards to the number of participants, the group responsible for the work process decided that all teams were to be full stack teams.
The advantage of this is that each group contains the capabilities to make an issue from the project backlog.
Based on the old reports from the last few semesters, that had been working on this project, the main problem was that one group would be waiting on another group to finish before their work could begin.
Another problem that earlier semesters faced were that some groups did not feel as if they were a part of the project, since they would only be working on their own little niche part of the project without interacting with the rest of the semester.

This new approach lead to the responsibilities of each group to be divided as can be seen on Figure~\ref{TBL:GroupResponsibility}.

\begin{table}[H]
\centering
\begin{tabular}{|l|l|}
\hline
\textbf{Group} & \textbf{Responsibility} \\ \hline
SW602F19 & Process\\ \hline
SW608F19 & Full stack\\ \hline
SW609F19 & Full stack\\ \hline
SW610F19 & Product Owner  \\ \hline
SW611F19 & Full stack \\ \hline
SW612F19 & Full stack \\ \hline
SW613F19 & Full stack \\ \hline
\end{tabular}
\captionof{table}{Table showing the different responsibilities across the groups}\label{TBL:GroupResponsibility}
\end{table}

This split was made, to make sure that every group was involved with every aspect of the project work.

\section{Tools}

\section{Project start up}
Before the first sprint could start the two groups responsible for both process and costumer contact where given time to both formalise the common work process and to contact the costumers.
Once both groups where ready they were to present their work to the rest of the groups.

Once all groups were aware of the work process and the backlog of the project the first sprint could be planned and stated as described in Part~\ref{PART:Sprints}.
