% !TeX root = ../main.tex
% This chapter is aimed to introduce both the subject of the project as well as our motivation for the project.
% It also aims to establish the initial problem statement, which the analysis will be conducted based on.
% This section should describe the basis of our project.
% really where our solution originates, based on a real world problem,
% even though our project is based on a "fun" project, it should still be possible to abstract it into real world applications
\chapter{Introduction}
This project is very different form any thing we have ever worked on doing our studies. 
The introduction there fore has be split into different sections, each describing the different aspects of the project. 
First an short section about the project it self and its motivation. 
Later an section about the start up process of the project work. 

\section{About the Giraf project}
The Giraf (Graphical Interface Resource for Autistic Folk) project is a software solution for autistic folk with little to non verbal communication.
The software solution is primarily used as a scheduling assistant, but the solution also aims to assist in education and games. 
The project has been developed by 6\textsuperscript{th} semester Software engineering students since 2011 and has changed a lot over the years. 
After each project period the project is passed along to the next 6\textsuperscript{th} semester software engineering students.

Since the project has been running since 2011, the semester coordinator Ulrik Mathias Nyman  has had the overview of the project since the beginning. 
He has been very engaged in the start up of the project at each semester in order to facilitate a smooth transition from the "normal" semester projects that the students has been used to, to this new multi group project work. 
He has throughout the project been facilitating meetings and contact with the faculty to make the work for the students as effortless as possible.

The project is an excellent learning opportunity for the students to learn that working with multiple costumers will result in unforeseen problems along the way.

Throughout the project that has been multiple costumers:

\begin{itemize}
	\item Børnehaven Birken, the local kindergarten
	\item Egebakken, the local school
	\item Fagcenter for Autisme og ADHD, a local center that facilitates activities for people with ADHD or autism
	\item IT-chef i Ældre- og Handicapforvaltningen, the IT chief in the Elderly- and Disability-management
\end{itemize}

These costumers will play a large role doing the project period since it is their needs that will be inserted into the project backlog. 
Hence all work that is done on the project will be done to facilitate their needs/wishes.

\section{Scrum of Scrums}

\section{Semester roles}
Since the project where so large with regards to the number of participants, the group responsible for the work process decided that all teams where to be full stack teams. 
The advantage of this is that each group contains the capabilities to make an issue from the project backlog. 
Based on the old reports from the last few semesters that had been working on this project, the main problem where that one group would be waiting on another groups to finish before their work could begin. 
Another problem that earlier semesters faced where that some groups did not feel as if they where a part of the project, since they would only be working on they own little niche part of the project without interacting with the rest of the semester.

This new approach lead to the responsibilities of each group to be divided as can be seen on Figure~\ref{TBL:GroupResponsibility}.

\begin{table}[H]
\centering
\begin{tabular}{|l|l|}
\hline
\textbf{Group} & \textbf{Responsibility} \\ \hline
SW602F19 & Process\\ \hline
SW608F19 & Full stack\\ \hline
SW609F19 & Full stack\\ \hline
SW610F19 & Product Owner  \\ \hline
SW611F19 & Full stack \\ \hline
SW612F19 & Full stack \\ \hline
SW613F19 & Full stack \\ \hline
\end{tabular}
\captionof{table}{Table showing the different responsibilities across the groups}\label{TBL:GroupResponsibility}
\end{table}

This split where to make sure that every group where involved with every aspect of the project work.

\section{Tools}

\section{Project start up}
The project group was formed with a common interest as the primary reason. 
All group members where interested in maintaining and upgrading the projects server infrastructure to accommodate the needs of the other project groups and the users of the system. 
Before any coding or optimising could be done, the different responsibilities for each group must be established. 
An initial meeting where held there all the groups and the semester coordinator had a brief discussion about the state of the project from last year. 

One group choose to take on the role as product owner and where to establish a relation with the costumers as quickly as possible. 
Once the costumer relation was established the requirements for the system could be established and the work distributed across re remaining groups.

The group responsible for developing the development model that all groups must work according to decided that the entire project should work with the Scrum of Scrums model.
While this model was used each group should be able to function as a full stack team, that is capable of handling every thing related to solving an issue in the project.
Hence the groups where assigned as follows:

After the responsibilities where handed out to the different project groups the actual development could begin.
The following project work will be described in Part~\ref{PART:Sprints}.