% !TeX root = ../main.tex

% This chapter is aimed to introduce both the subject of the project as well as our motivation for the project.
% It also aims to establish the initial problem statement, which the analysis will be conducted based on.
% This section should describe the basis of our project.
% really where our solution originates, based on a real world problem,
% even though our project is based on a "fun" project, it should still be possible to abstract it into real world applications
\chapter{Introduction}

This project is on a very different form than anything we have previously worked on during our studies.
The introduction has therefore been split into different sections, each describing the different aspects of the project.
First a short section about the project itself and its motivation followed by a section about how the process work should be conducted and ending with a section about the start up process of the project period.

\section{About the Giraf project}
The Giraf (Graphical Interface Resource for Autistic Folk) project is a software solution for autistic folk with little to no verbal communication.
The software solution is primarily used as a scheduling assistant, but the solution also aims to assist in education and games.
The project has been developed by 6\textsuperscript{th} semester Software engineering students since 2011 and has changed a lot over the years since its initiation.
When a project period is completed, the project is passed along to the following year's 6\textsuperscript{th} semester software engineering students.

The semester coordinator Ulrik Nyman has had the overview of the project since its beginning in 2011.
Ulrik Nyman expressed a lot of engagement in the start up of the project in order to facilitate a smooth transition from the usual type of semester projects that the students have been working in, to this new multi group project work.
Ulrik Nyman also facilitated meetings and contact with the faculty at AAU to make the work for the students as effortless as possible.

The Giraf project is an obvious learning opportunity for the students to experience how working with multiple costumers will result in unforeseen problems along the way.

Throughout the project there have been multiple costumers:

\begin{itemize}
	\item Børnehaven Birken (a local special needs kindergarten)
	\item Egebakken (a local special needs school)
	\item Fagcenter for Autisme og ADHD (a part of the municipality that facilitates activities for people with ADHD or autism)
	\item IT-chef i Ældre- og Handicapforvaltningen (the IT department manager in the Elderly- and Disability-management)
\end{itemize}

These costumers play a large role during the entire project, since it is their needs/wishes that the project backlog will be based upon.

\section{Scrum of Scrums}
Since this project is a multi-group semester-wide project, a project-group was made responsible for managing the work process.
This group decided that the entire project should be run as a scrum project, but the individual groups were to decide how they would work internally.

Another group was made responsible for facilitating the contact with the costumers, hence becoming the product owners of the project.
The product owner group are also responsible for planning each sprint such that the costumers would get the highest amount of value as fast as possible.
All groups are also to participate in the planning of each sprint, as only the groups know how much time they individually have for working on the project, and thereby an estimate of how many story points they can complete during a sprint.

During a sprint each group sends one representative to a weekly scrum stand up, such that every group know what/how the other groups are doing.
After each sprint a sprint retrospective and review meeting is held in order for the groups to share their experiences from the now completed sprint, which will then give beneficial experiences for the next sprint period.
The outcome of this meeting is the points to make better in the next sprint period.

In our group we followed the waterfall development method. 
This is to ensure that the issues assigned to the group would be completed in time for release. 
This method was chosen because of the teams relatively small size compared to the other groups and because it was deemed unlikely that the product owner group would make any changes to an issue once it has been added to a sprint.
From the perspective of our group the process is actually a waterfall of scrums, whereas for the rest of the project groups the scrum of scrums method are more descriptive.
So for the rest of this report the process for the entire project will simply be called scrum of scrums.

\subsection{Crunch}
As a part of the scrum of scrums process, the process group wanted a release preparation where all groups were gathered in order to fix bugs discovered during the release period. 
During this release preparation all groups are supposed to work on the same set of issues until the product is deemed stable enough for release. 
This step is to ensure that the released product at the end of each release cycle is as stable as possible since the main objective of this project period is to ensure stability.

\section{Semester roles}
Since the project is so large with regards to the number of participants, the group responsible for the work process decided that all teams were to be full-stack teams.
The advantage of this is that each group contains the capabilities to complete all apsects of an issue from the project backlog.
Based on the old reports from the last few semesters who had been working on this project, the main problem was that one group would be waiting on one or more other groups to finish their work, before they could begin on their own task.
Another problem earlier semesters faced was that some groups did not feel as if they were a part of the project, since they would only be working on their own little niche part of the project without interacting with the rest of the semester groups.

This new approach lead to the responsibilities of each group to be divided, as can be seen on Figure~\ref{TBL:GroupResponsibility}.
\begin{table}[H]
\centering
\begin{tabular}{|l|l|}
\hline
\textbf{Group} & \textbf{Responsibility} \\ \hline
SW602F19 & Scrum\\ \hline
SW608F19 & Full stack\\ \hline
SW609F19 & Full stack\\ \hline
SW610F19 & Product Owner  \\ \hline
SW611F19 & Full stack \\ \hline
SW612F19 & Full stack \\ \hline
SW613F19 & Full stack \\ \hline
\end{tabular}
\captionof{table}{Table showing the different responsibilities across the groups}\label{TBL:GroupResponsibility}
\end{table}

\section{Meta Groups}\label{SEC:MetaGroups}
To facilitate the full stack groups, the addition of meta groups was made.
These groups will have the function of facilitating the full stack work so the groups have a better connection, cooperation and a common understanding of the current progress of the project.
There were three meta groups introduced in the semester:

The \textit{frontend} meta group is responsible for everything regarding the user interface; they will be taking decisions for all of the groups in regards to the user interface design.

The \textit{backend} meta group is the responsible party when it comes to all the functionality that is hidden away behind the user interface, and the decisions that go along with it.

The \textit{server} meta group handles the required decisions on server infrastructure, and manages the aquisition of servers from AAU.

A group member was assigned to each of the meta groups in order to keep track of what is going on in each of the project groups. 
Meetings are held to keep each meta group informed on what is currently happening across the entire project. 
these meetings will also be used to make concrete decisions in regard to changes affecting the entire Giraf project.

\section{Tools}
As a part of this project a number of different tools was used to facilitate the development done by the different groups.
In the following each of the different tools will be described. 

\subsection{GitHub}
For hosting the code of the Giraf project the previous semesters has been using GitLab in a self hosted environment. 
It was chosen to migrate from the self hosted GitLab to the regular GitHub.
This was primarily done so project did not have to maintain the GitLab instance and servers.
The GitHub has setup as an organisation so that all the different members of the project would add a team to the organization. On \autoref{GitHubOrg} the front page of the organization can been seen. 

\figur{1}{images/GitHub.png}{GitHub Giraf organization main page}{GitHubOrg}

Each developer group where added as GitHub Teams as can be seen on \autoref{GitHubTeams} so it would be easier to assign the different groups to a given issue or simply to tag the entire group in a comment. 

\figur{1}{images/GitHubTeams.png}{GitHub Giraf team overview}{GitHubTeams}

\subsection{Azure Pipeline}
With GitHub it is possible to integrate different solutions for testing that a given branch where up to the quality standard. 
Before each branch is able to merge it needs to pass two criteria. 
First it needed to be review by two developers and one representative from the product owner group. 
Additionally all checks implemented in the Azure pipeline would need to pass, as can be seen on \autoref{GitHubMergeControl}. 

\figur{1}{images/GitHubMergeControl.png}{GitHub merge control}{GitHubMergeControl}

The checks implemented where both the Flutter linter that would check that the code design followed Flutter guidelines. 
The pipeline would also check if the flutter compile could compile with out any warnings/errors. 
On \autoref{GitHubPipeline} the build result of a pipeline run can be seen.

\figur{1}{images/GitHubPipeline.png}{GitHub Azure Pipeline integration}{GitHubPipeline}


\subsection{Docker}

\subsection{Slack}

\subsection{Email}


\section{Project start up}
Before the first sprint could start the two groups responsible for both process and costumer contact were given time to both formalise the common work process and to contact the costumers.
Once both groups where ready they were to present their work to the rest of the groups.

Once all groups were aware of the work process and the backlog of the project the first sprint could be planned and stated as described in Part~\ref{PART:Sprints}.
