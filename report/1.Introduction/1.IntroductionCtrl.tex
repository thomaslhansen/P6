% !TeX root = ../main.tex
% This chapter is aimed to introduce both the subject of the project as well as our motivation for the project.
% It also aims to establish the initial problem statement, which the analysis will be conducted based on.
% This section should describe the basis of our project.
% really where our solution originates, based on a real world problem,
% even though our project is based on a "fun" project, it should still be possible to abstract it into real world applications
\chapter{Introduction}
This project is very different form any thing we have ever worked on doing our studies. 
The introduction there fore has be split into different sections, each describing the different aspects of the project. 
First an short section about the project it self and its motivation. 
Later an section about the start up process of the project work. 

\section{About the project}
The project for this semester is an multi group project where the primary goal is to deliver some value to the end users/costumers. 
The project is named Giraf and is a tablet environment aimed at autistic children with little or no verbal communication.
The project has been developed by Software 6. student since 2011 and has changed a lot over the years. 
The Giraf project is used as a tool in many institutions in the Danish healthcare that specialises with autism. 

The primary goal for the project over the last few years has been to make the entire system more reliable and stable. 
Which has been done with great success but at the cost of all the applications in the project has stop working since the original back end was changed last year. 
To further complicate the project the system was originally only available for Android which many of the users does not use any more since a large shift in technology has been done in the healthcare system so that the system should now be compatible with both Android and IOS devices. 


\section{Project start up}
The project group was formed with a common interest as the primary reason. 
All group members where interested in maintaining and upgrading the projects server infrastructure to accommodate the needs of the other project groups and the users of the system. 
Before any coding or optimising could be done, the different responsibilities for each group must be established. 
An initial meeting where held there all the groups and the semester coordinator had a brief discussion about the state of the project from last year. 

One group choose to take on the role as product owner and where to establish a relation with the costumers as quickly as possible. 
Once the costumer relation was established the requirements for the system could be established and the work distributed across re remaining groups.

The group responsible for developing the development model that all groups must work according to decided that the entire project should work with the Scrum of Scrums model.
While this model was used each group should be able to function as a full stack team, that is capable of handling every thing related to solving an issue in the project.
Hence the groups where assigned as follows:

\begin{table}[H]
\centering
\begin{tabular}{|l|l|}
\hline
\textbf{Group} & \textbf{Responsibility} \\ \hline
SW602F19 & Scrum\\ \hline
SW608F19 & Full stack\\ \hline
SW609F19 & Full stack\\ \hline
SW610F19 & Product Owner  \\ \hline
SW611F19 & Full stack \\ \hline
SW612F19 & Full stack \\ \hline
SW613F19 & Full stack \\ \hline
\end{tabular}
\end{table}

After the responsibilities where handed out to the different project groups the actual development could begin.
The following project work will be described in Chanter~\ref{CAP:PrintsOfTheDevelopment}.