\subsection{User Stories} \label{ssec:UserStories}
Even though the entire Giraf-project is worked on by multiple groups, each individual group will do mostly independent work.
A method for dividing workloads between the groups was therefore needed, which would ensure that each group had a discrete part of the Giraf-project to work on, and with as little overlap with the work of other groups as possible.

The method chosen was to assign each project group with one or more \textit{user stories} devised from the customers.
These user stories would contain a requested feature or perhaps an error the customer might have come across.

By having a product-owner group discussing these requests and thoughts with the customers, it was possible to make measurable goals for the groups to work towards.
Measurability is important, as this gives a precise definition of when the requested feature (or error-fix) has been completed to a satisfactory level.

The user stories were written in a natural language, such that the story described the desired feature or a problem be in a feature that was already implemented.
On GitHub we used the issue feature to keep track of these user stories.
The issues on GitHub were often expanded, compared to the user story, with more detail.
Some examples of expansions would be suggested solutions to the issue, other issues that the issue depended on, and/or some pictures from the design guide.

During this report, multiple of these user stories will be addressed, instead of a single and more extensive problem statement.
The user stories assigned to our group throughout the project will be presented, as well as our approach to how the expected goals will be met.
