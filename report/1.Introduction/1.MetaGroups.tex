\section{Meta groups}\label{SEC:MetaGroups}
To facilitate the full stack groups, the addition of meta groups was made.
These groups will have the function of facilitating the full stack work so the groups have a better connection, cooperation and a common understanding of the current progress of the project.
There were three meta groups introduced in the semester:

The \textit{frontend} meta group is responsible for everything regarding the user interface; they will be taking decisions for all of the groups in regards to UI design.

The \textit{backend} meta group is the responsible party when it comes to all the functionality that is hidden away behind the UI, and the decisions that go along with it.

The \textit{server} meta group handles the required decisions on server infrastructure, and manages the aquisition of servers from AAU.

A group member was assigned to each of the meta groups in order to keep track of what is going on in each of the project-groups. 
Meetings are held to keep each meta group informed on what is currently happening across the entire project. 
these meetings will also be used to make concrete decisions in regard to changes affecting the entire Giraf project.
