\section{Meta groups}
To facilitate these full-stack groups, the addition of meta groups was made.
The main reason the meta-groups were formed was that in the previous semesters some groups felt left out.
These groups will have the function of facilitating the full-stack work so the groups have a better connection, cooperation and a common understanding of the current progress of the project.
There were three meta groups introduced in the semester:

The \textit{frontend} meta group is responsible for everything regarding the user interface, they will be taking decisions for all of the groups in regards to UI design.

The \textit{backend} meta group will be the responsible party when it comes to all the functionality that is hidden away behind the UI, and the decisions that go along with it.

The \textit{server} meta group take the required decisions about server infrastructure, and manages the aquisition of servers from AAU.

A group member was assigned to each of the meta groups in order to keep track of what was going on in each of the groups. Meetings will be held to keep each group informed about what is currently happening in the project, across all of the groups. These meetings will also be used to make concrete decisions in regard to changes affecting the entire Giraf project.