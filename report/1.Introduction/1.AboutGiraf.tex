\section{About the Giraf project}
The Giraf (Graphical Interface Resource for Autistic Folk) project is a software solution for autistic folk with little to no verbal communication.
The software solution is primarily used as a scheduling assistant, but the solution also aims to assist in education and games.
The project has been developed by 6\textsuperscript{th} semester Software engineering students since 2011 and has changed a lot over the years.
After each project period the project is passed along to the next 6\textsuperscript{th} semester software engineering students.

Since the project has been running since 2011, the semester coordinator Ulrik Mathias Nyman  has had the overview of the project since the beginning.
He has been very engaged in the start up of the project at each semester in order to facilitate a smooth transition from the "normal" semester projects that the students has been used to, to this new multi group project work.
He has throughout the project been facilitating meetings and contact with the faculty to make the work for the students as effortless as possible.

The project is an excellent learning opportunity for the students to learn that working with multiple costumers will result in unforeseen problems along the way.

Throughout the project there has been multiple costumers:

\begin{itemize}
	\item Børnehaven Birken, the local kindergarten
	\item Egebakken, the local school
	\item Fagcenter for Autisme og ADHD, a local center that facilitates activities for people with ADHD or autism
	\item IT-chef i Ældre- og Handicapforvaltningen, the IT chief in the Elderly- and Disability-management
\end{itemize}

These costumers will play a large role during the project period since it is their needs that will be inserted into the project backlog.
Hence all work that is done on the project will be done to facilitate their needs/wishes.