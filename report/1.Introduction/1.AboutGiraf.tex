\section{About the Giraf project}
The Giraf (Graphical Interface Resource for Autistic Folk) project is a software solution for autistic folk with little to no verbal communication.
The software solution is primarily used as a scheduling assistant, but the solution also aims to assist in education and games.
The project has been developed by 6\textsuperscript{th} semester Software engineering students since 2011 and has changed a lot over the years since its initiation.
When a project period is completed, the project is passed along to the following year's 6\textsuperscript{th} semester software engineering students.

The semester coordinator Ulrik Nyman has had the overview of the project since its beginning in 2011.
Ulrik Nyman expressed a lot of engagement in the start up of the project in order to facilitate a smooth transition from the usual type of semester projects that the students have been working in, to this new multi group project work.
Ulrik Nyman also facilitated meetings and contact with the faculty at AAU to make the work for the students as effortless as possible.

The Giraf project is an obvious learning opportunity for the students to experience how working with multiple costumers will result in unforeseen problems along the way.

Throughout the project there have been multiple costumers:

\begin{itemize}
	\item Børnehaven Birken (a local special needs kindergarten)
	\item Egebakken (a local special needs school)
	\item Fagcenter for Autisme og ADHD (a part of the municipality that facilitates activities for people with ADHD or autism)
	\item IT-chef i Ældre- og Handicapforvaltningen (the IT department manager in the Elderly- and Disability-management)
\end{itemize}

These costumers play a large role during the entire project, since it is their needs/wishes that the project backlog will be based upon.
