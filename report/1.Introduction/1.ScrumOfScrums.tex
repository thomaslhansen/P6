\section{Scrum of Scrums}
Since this project is a multi group, semester wide project, a team was made responsible for managing the work process.
This group decided that the entire project should be run as a scrum project, but that the individual groups were to decide how they would work.

While one group was responsible for the work process, another group was responsible for facilitating the contact with the costumers, hence becoming the product owners of the project.
They were also responsible for planning each sprint so that the costumers would get the most amount of value as fast as possible.
All groups were also participating in the planning of each sprint, as only they knew how much time they had for project work and an estimate of how many story points they could complete during the sprint.

During a sprint each group were to send one representative to a weekly scrum stand up so that all the groups knew what each group was doing.
After each sprint a sprint retrospective and review meeting were held in order for the groups to share their experiences from the last sprint for the next sprint period.
The outcome of this meeting were the points to make better in the next sprint period.

In our group we would also follow the scrum guidelines in order to produce a better product with a higher degree of agility in anticipation of changing requirements from the costumers.
Hence the name Scrum of Scrums, since we had multiple scrums processes running independent of each other.