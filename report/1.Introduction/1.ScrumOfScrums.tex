\section{Scrum of Scrums}
Since this project is a multi-group semester-wide project, a project-group was made responsible for managing the work process.
This group decided that the entire project should be run as a scrum project, but the individual groups were to decide how they would work internally.

Another group was made responsible for facilitating the contact with the costumers, hence becoming the product owners of the project.
The product owner group are also responsible for planning each sprint such that the costumers would get the highest amount of value as fast as possible.
All groups are also to participate in the planning of each sprint, as only the groups know how much time they individually have for working on the project, and thereby an estimate of how many story points they can complete during a sprint.

During a sprint each group sends one representative to a weekly scrum stand up, such that every group know what/how the other groups are doing.
After each sprint a sprint retrospective and review meeting is held in order for the groups to share their experiences from the now completed sprint, which will then give beneficial experiences for the next sprint period.
The outcome of this meeting is the points to make better in the next sprint period.

In our group we follow the waterfall development method. 
This is to ensure that the issues assigned to the group would be completed in time for release. 
This method was chosen because of the teams relatively small size compared to the other groups and because it was deemed unlikely that the product owner group would make any changes to an issue once it has been added to a sprint.
From the of our group the process is actually a waterfall of scrums, whereas for the rest of the project groups the scrum of scrums method are more descriptive.
So for the rest of this report the process for the entire project will simply be called scrum of scrums.

\subsection{Crunch}
As a part of the scrum of scrums process, the process group wanted a release preparation where all groups where gathered in order to fix bugs discovered during the release period. 
During this release preparation all groups are supposed to work on the same set of issues until the product is deemed stable enough for release. 
This step is to ensure that the released product at the end of each release cycle is as stable as possible since the main objective of this project period is to ensure stability.
