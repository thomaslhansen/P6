\section{Scrum of Scrums}
Since this project is a multi-group semester-wide project, a project-group was made responsible for managing the work process.
This group decided that the entire project should be run as a scrum project, but the individual groups were to decide how they would work internally.

Another group was made responsible for facilitating the contact with the costumers, hence becoming the product owners of the project.
The product owner group are also responsible for planning each sprint such that the costumers would get the highest amount of value as fast as possible.
All groups are also to participate in the planning of each sprint, as only the groups know how much time they individually have for working on the project, and thereby an estimate of how many story points they can complete during a sprint.

During a sprint each group sends one representative to a weekly scrum stand up, such that every group know what/how the other groups are doing.
After each sprint a sprint retrospective and review meeting is held in order for the groups to share their experiences from the now completed sprint, which will then give beneficial experiences for the next sprint period.
The outcome of this meeting is the points to make better in the next sprint period.

In our group we will also follow the scrum guidelines. This is to produce a better product with a higher degree of agility in anticipation of changing requirements from the costumers.
Hence the name Scrum of Scrums, since we have multiple scrum processes running independent of each other.
