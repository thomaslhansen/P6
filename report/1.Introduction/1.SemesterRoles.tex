\section{Semester roles}
Since the project is so large with regards to the number of participants, the group responsible for the work process decided that all teams were to be full stack teams.
The advantage of this is that each group contains the capabilities to complete all aspects of an issue from the project backlog.
Based on the old reports from the last few semesters who had been working on this project, the main problem was that one group would be waiting on one or more other groups to finish their work, before they could begin on their own task.
Another problem earlier semesters faced was that some groups did not feel as if they were a part of the project, since they would only be working on their own little niche part of the project without interacting with the rest of the semester groups.

This new approach lead to the responsibilities of each group to be divided, as can be seen on Figure~\ref{TBL:GroupResponsibility}.
\begin{table}[H]
\centering
\begin{tabular}{|l|l|}
\hline
\textbf{Group} & \textbf{Responsibility} \\ \hline
SW602F19       & Process                 \\ \hline
SW608F19       & Full stack              \\ \hline
SW609F19       & Full stack              \\ \hline
SW610F19       & Product Owner           \\ \hline
SW611F19       & Full stack              \\ \hline
SW612F19       & Full stack              \\ \hline
SW613F19       & Full stack              \\ \hline
\end{tabular}
\captionof{table}{Table showing the different responsibilities across the groups}\label{TBL:GroupResponsibility}
\end{table}
