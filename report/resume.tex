\chapter*{Resumé}
% Intro to Giraf
The Giraf (Graphical Interface Resource for Autistic Folk) project is a mobile software solution.
It is made to assist autistic folk with little to no verbal communication ability.
Giraf is made to be a collection of mobile applications that will assist autistic folk in their everyday lives
The only working application in this collection is called 'Weekplanner', an application for scheduling an autistic child's day using pictograms.

% Intro to the Giraf working model
Giraf's working model is based on the agile development Scrum.
It did start out as a Scrum of Scrums, but during the project parts of scrum were abandoned.
The individual groups that worked on Giraf were also not required to use scrum.
Hence the Giraf project is only loosely based on the Scrum method, where Giraf uses most of the elements the individual groups tried different working methods internally, e.g., this group used the Waterfall model.

% Weekplanner language and framework change

% Server stability (Docker Swarm)

% WEB-API pictogram performance update

% General state of Giraf after the semester

\newpage
