\section{Server}

Throughout the project period the overall structure of the project has changed a lot. 
With the new server layout, most of the way the app communicates with the API has changed. 
On \autoref{FIG:OldGirafOverview} and \autoref{FIG:NewGirafOverview} both the new and the old overview can be seen. 

The key difference is that the app can now be compiled to both IOS and Android, which was previously not possible. 
The old app was developed in Xamarin since it, at that time, could be compiled to IOS but after the release of IOS version 10, the compiler no longer worked. 
Since the new app is written in Flutter, it is able to compile to both devices. 

\figur{0.55}{images/GirafOverviewOld.png}{Diagram showing the old server overview}{FIG:OldGirafOverview}

Another key difference is that the database is no longer accessible from the internet since it is only connected to the API network inside the Docker Swarm. 
Even the API's are no longer directly connected to the internet since they are served via the reverse NGINX proxy.

On the earlier version of the API it would connect to the database via the public internet even though they where connected to the same internal network.
This resulted in a lot of traffic being transmitted via the internet even through a more direct route was possible. 

\figur{1}{images/GirafOverview.png}{Diagram showing the new server overview}{FIG:NewGirafOverview}

Finally the amount of both API and proxy services has been increased to guarantee stability and high availability of the services. 
With the old layout, if the API were to crash the API had to be manually restated which was a huge pain for the server group, in terms of additional work and stress. 
With the new layout however, the Docker Swarm has been setup to conduct health checks on the services and automatically restart the service in case of a failure.
