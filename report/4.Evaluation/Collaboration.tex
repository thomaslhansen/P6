\chapter{Collaboration}
The Giraf project has been a collaborative project where multiple groups were to work on the same existing codebase.
Throughout the project period, collaborative work between all groups were conducted in various ways.  
In the following sections this work will be both described and evaluated from our point of view.

\section{Project Groups}
The project was conducted by seven full stack groups, where each group had the capabilities to conduct all types of work.
A key component in order to work in this multi group environment was communication.
Most of the collaborative communication has been conducted via online channels.
The PO group has been in charge of assigning issues to the groups, where Slack and GitHub have both been heavily utilised.

After the PO group had made the user stories into concrete issues, and potentially split into multiple issues if the original issue was large enough, the PO group could then assign a group to the issue on GitHub.
For all issue related communication, GitHub comments was chosen as the primary way of communication. 
This was chosen in case an issue was not completed, as all communication would be saved for the next group, whom were to complete the issue.
Once assigned to an issue, the group members would get an email, from GitHub, and the PO group would also notify the group on Slack. 

Most groups were available on Slack during normal working hours and could therefore be contacted via Slack.
However, since the group rooms for the groups were in such a close proximity, the communication between groups often ended up being verbal.
Thus the verbal communication often ended up being faster than going through official channels for everyday communications.
However, important messages affecting all groups, should of course use the official channels whenever possible.

When a change had been made, and a pull request had been issued, someone would more often than not come and ask the other group before the Slack message was out, as this online form of communication could sometimes be a little slow.
Once a pull request had been made, the process group were responsible for assigning the review task to one or more groups. 
This was done by assigning the group members on GitHub and notifying them via Slack. 
Once assigned, the group being reviewed would contact the other groups in order to make the entire review process as quick as possible.
This increased efficiency came from nudging the review into a higher priority than what was currently being worked on, as well as being available for any and all questions the reviewers might have.

Larger pieces of information, like changes to standards and the like, were still sent out through Slack, as those changes were usually documented on the GitHub Wiki.
Whenever a change in either the process or standards were made, the documentation on the GitHub Wiki would be updated. 
The groups might be notified about the change on Slack, if not, at the next Scrum of Scrums stand up meeting.

Since this semester is the first to implement full stack teams there was no previous experience to lean against. 
However, it has, for most people, been a pleasant experience since the work has not been solely on one part of the project.
We would highly recommend full stack teams for future semesters.

\section{Meta Groups}
As described \autoref{SEC:MetaGroups} meta groups were introduced into the project to support the full stack teams.
Throughout the project period, these meta groups has been used in various ways as described in \autoref{PART:Sprints}. 

During the meta group meetings, all groups would be represented and would have a say in any decisions made during the meeting.
This was not only to ensure that the correct solution was chosen, but also to make sure that they had an influence on the decision.
As a side effect hereof, every group would always be informed about the status of any issues (and possible solutions) affecting all of the front end, and being able to share this information within their own group.

Whenever a large, critical issue arose, a gathering of people with the relevant knowledge would be assembled in order to implement a solution for the issue. 
This would often be faster than a single full stack group and would result in the entire system being operational as well as following the meta groups' suggested standards for the project.
  
The meta group meetings became less frequent as a result of the standards being set, and would, at the end of the project period, result in some of the meta groups being inactive.
The overall assessment of the meta groups were mainly positive, as described in \autoref{SEC:Sprint4Retrospective}, and it is highly recommended for future semesters.
