\Section{Collaboration}
In this semester.
The Giraf project has as discussed in previous chapters been a collaborative task.
Most of the collaborative effort, has been handled via online channels.
The process group has been in charge og assigning work related tasks to groups, where Slack \& github have both been heavily utilised.

After the PO group had made the user stories into something tangible, and potentially multiple tasks, the process group could then assign a group to a corresponding issue on Github.
The communication for these assignments, as well as other information which people could be in need of, could be inquired about on Slack.
Most groups were available on Slack during most of the day, and could therefore be contacted on slack, even if they were not available at the univercity that day.

Realistically speaking however, since the group rooms for all the students have been in such a close proximity.
The communication between groups very often ended up being face to face.
When a change had been made, and a pull request had been issued.
Someone would more often than not come and ask the other group before the slack message was out, as this form of communication could sometimes be a little slow.


