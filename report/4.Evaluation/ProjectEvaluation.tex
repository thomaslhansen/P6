As the text written in \autoref{CAP:ProjectOverview} focuses on the state of the product (and accompanied documentation), this chapter will give an evaluation on the project work done internally by this group, both in terms of working with the product (designing and coding) and in terms of the documentation in the report.

\section{Product Development}
Developing this semester has been quite different, compared to earlier semesters, as each of the different issues worked on were not necessarily in immediate succession of each other.
The first sprint focused primarily on leveling the playing field for every member of the Giraf organization by migrating from Xamarin Forms to Flutter, and thereby allowing members to develop optimally on a Linux based operating system.
Any internal product development within the group could therefore be seen as preliminary work, i.e. getting familiar with the new language and framework: Dart and Flutter.


The second and fourth sprint focused on a user interface issue, which meant working within our newly adopted language and framework.
We experienced both successful implementation, as well as a forfeit to solve an issue ourselves, followed by a helping hand from other groups.
In the end, both were suitable ways to further the development of the product.

The third sprint revolved around server work, which accounted for the bulk of the product development contributed by this group.
It also accounted for the most work done by more or less just one group member.
While this initially would not be a preferable way of working in a group, we attempted to compensate by internally sharing the knowledge used to implement the server work through the documentation in the report.
Despite this compensation, the optimal work flow would have been to include more group members in the loop, even though this would most likely affect the efficiency negatively.

\section{Report Development}
Working on the report has, throughout the entire semester, been a recollection process, e.g. while product development was in sprint three, the group was writing on the report for sprint two.
This seemed to work well for us, as there always was something to write for any member not working with product development at a given time.

We managed to mostly stay on track with writing about the previous sprint during the current sprint, and only minor parts fell through this goal.
These minor parts were easy to pick up and complete several of them within half a day.

As described in \autoref{SEC:InternalCollabReport}, the group wrote all text in \LaTeX, and managed to steer clear of most writing collisions by utilizing GitHub as a version control tool.
Using the advantages of branching, reviewing and then merging, seems to be the best way this group could have handled the report development.
