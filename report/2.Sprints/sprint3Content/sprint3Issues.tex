\section{Issues}\label{SEC:Sprint3Issues}
During the planning period, the product owner group decided that since we had been the driving force of the new server layout, we should continue the work on the servers.
Hence our sprint planning period was largely specified beforehand.
The product owner group had assigned the group to two different issues on GitHub(\href{https://github.com/aau-giraf/wiki/issues/34}{Issue \texttt{\#}34} and \href{https://github.com/aau-giraf/wiki/issues/35}{Issue \texttt{\#}35})

\textbf{Issue 34 - Migrate the WEB-API to Docker and document the work} was as the name suggests that the old WEB-API should be migrated to work in the new Docker Swarm environment.
This had not previously been possible since the WEB-API had a dependency at compile time.
This dependency included the database connection string in plaintext and if pushed to the Docker Hub it could be reverse engineered and the password could be recovered.
One of the main objectives for this issue is therefore to remove the dependency of this appsettings file.

\textbf{Issue 35 - Migrate the MySQL Database to Docker and document the work} was an critical issue for the project since the existing database had been hosted in such a way that the database was accessible from the public internet.
This is a serious security risk since the MySQL version 5.7.11 has not been updated to mitigate any security issues found during the lifetime of the database.
As a result of this, the database should be migrated from the existing server to the new servers, since the new servers do not have the MySQL port $3306$ accessible from the internet.
