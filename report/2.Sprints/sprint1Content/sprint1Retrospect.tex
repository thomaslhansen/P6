\section{Retrospective}
The retrospective meeting for sprint one gave better insight into what everyones view of the process was so far and suggestions for changes that could potentially improve the process, as to make everyone work more effective, if possible to do so. 

To get the discussion started on what the good and bad parts of the process was so far we split up, such that no-one were discussing with someone they were in a project group with. 
This discussion gave us some insight into how the other groups saw the process and what they liked/disliked about it. 

Some of the PO group felt like they often did not have time to develop anything, as they were often busy with making the user stories and communicating with the customers. 
The full stack groups were mostly satisfied with the process, it was mostly that they were not certain of what to do when a user story was very vague or broad, the way to deal with this was to talk with the PO group and either get some clarification on the user story or get help to divide it into smaller user stories that could then be picked seperately.

Once we were done discussing we put up the suggestions for improvements to \href{www.dotstorming.com}{dotstorming}, and then we each, individually, placed our votes on our top two suggestions.
This way it was easy to get an overview of what most people wanted to make improvements upon.

The top five suggestions for improvements were:
\begin{enumerate}
  \item Technical explanations of the user stories.
  \item Fewer meetings on days without lectures.
  \item Consider discarding the sprint planning.
  \item More hack-a-thons.
  \item User stories should be more describing.
\end{enumerate}

The top suggestion was about making some comments with technical insight into the user stories to help others get a better idea of whether they were up for the task.
These comments would come from anyone with the time and knowledge to give some quick technical insight that could help others get started. 

Some groups prefered to work from home whenever possible, so they did not feel it was ideal to have meetings across groups on days that did not have lectures. 
This was mainly targeted at the scrum meetings, such as stand-up, sprint review and sprint retrospective, as the meta groups usually agreed on dates internally.

There was a wish for more hack-a-thons, as there has not really been any and there is not many social activities across the groups yet. 
% I'm not certain of this, as I haven't actually paid it any attention

The 5\textsuperscript{th} suggestion very much relates to the 1\textsuperscript{st}, as this is more a request for more information about the user stories. 
Currently the user stories are very short and usually only one line, so there is a wish for more explanation as to what a user story is about exactly.
