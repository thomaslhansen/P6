The first sprint had the purpose of getting to know the codebase, how the whole project was structured, and work out any potential problems with the work process. 
The amount of actual work on the product was not expected to be as high as future sprints, as we were not familiar with the codebase yet and needed some time to learn how the system worked. 
The sprint was planned to start on 25\textsuperscript{th} of February and end on 18\textsuperscript{th} of March so a total of 16 workdays.

\section{Planning}
The sprint planning was held on the 25\textsuperscript{th} of February, and was done by having a meeting with the entire giraf organization present. 
The product owner group presented some user stories as well, as what the goal of the project was. 
The customers wanted more stability and reliability in the apps that already works, and did not care whether new and exciting functionality was added. 

After the presentation each group picked 3 user stories that they wanted to complete, even though the user stories was not necessarily a high priority for the project. 
The idea was that the groups got something they wanted to work with as a startup and would then later pick based on priority.

Then each group was given 1 of the 3 user stories that they had chosen, and the user stories were split into tasks, and it was estimated how long it would take to complete the user story. 
If we were uncertain of whether we would be able to complete the user story or it came close to the deadline, we were to only assign ourselves to this 1 user story, but if it was clear that another user story could be managed, we were allowed to take another.
but if it was clear that it could be managed another user story we were allowed to take more. 
The product owner wanted to be certain that the user stories, that were picked, were completed and not have unfinished user stories at the end of the sprint.

We were not certain of the time it would take to complete the user story that we had picked, especially as another group had a user story that potentially would conflict with some parts of our user story, hence this conflict had to be resolved.

Our user story was: "As a guardian I would like to confirm with a password that the system is changing to guardian mode so that a citizen cannot gain access to it".

The other group had the user story: "As a guardian, I would like that the app is fully available offline so that I can still use it if the internet is down"

So as our user story was regarding authentication of a guardian entering guardian mode and the others regarding making the entire app work offline, including login authentication, therefore a resolution to who would work on what had to be found, such that we did not get any merge conflicts in the login authentication. 
As our issue contained some frontend and backend while the others only really concerned the backend we agreed that they would resolve the backend work and we would work on the frontend part of our user story and then use the new login functionality that they made.

\section{Encountered issues}
During the sprint we encountered some issues with the framework xamarin, the framework that the frontend of the app was written in. 
About a third of the giraf organization were on the OS GNU/Linux and unable to compile; those that were able to compile could either not sign the apk file or were unable to install the app onto a tablet with the apk file.
So to resolve this issue the frontend meta group had a "emergency" meeting where the issue would be discussed.

\subsection{Xamarin framework issue}
The core of the issue was that the frontend of the app had been switched to the xamarin framework and this was not officially supported for GNU/Linux and there were no plans to make it officially supported \cite{xamarinSupport}.
%https://forums.xamarin.com/discussion/1599/xamarin-studio-in-linux 
%https://xamarin.uservoice.com/forums/144858-xamarin-platform-suggestions/suggestions/2697538-linux-support 
So to mitigate this for us and future work we wanted to see if it was a viable and desired option to switch to another framework that was supported on all OS, such that we would not have to use a virtual machine with Microsoft Windows or Mac OS to compile and test any changes, which would not be an optimal solution.

We first considered whether we should switch to another framework at all.
For this discussion the frontend meta group held a meeting, where all were invited to listen in, as this was a big decision to be made.
The meeting resulted in the decision to switch to a new framework called flutter, but before the decision was final those attending the meeting were to discuss it with their respective group and perhaps get some input that was not brought up during the meeting.
Now a final meeting was held to discuss how the process should be handled.
It was decided to handle the switch by letting a group of volunteers dedicate the weekend to write the core part of the frontend into flutter, and then to start everyone up the Monday would be used to make small/easy user stories that would be used to get started with flutter on Tuesday.
