The first sprint had the purpose of getting to know the codebase, how the whole project was structured, and work out any potential problems with the work process. 
The amount of actual work on the product was not expected to be as high as future sprints, as we were not familiar with the codebase yet and needed some time to learn how the system worked. 
The sprint was planned to start on 25\textsuperscript{th} of February and end on 18\textsuperscript{th} of March so a total of 16 workdays.

\section{Planning}
The sprint planning was held on the 25\textsuperscript{th} of February, and was done by having a meeting with the entire giraf organization present. 
The product owner group presented some user stories as well, as what the goal of the project was. 
The customers wanted more stability and reliability in the apps that already works, and did not care whether new and exciting functionality was added. 

After the presentation each group picked 3 user stories that they wanted to complete, even though the user stories was not necessarily a high priority for the project. 
The idea was that the groups got something they wanted to work with as a startup and would then later pick based on priority.

Then each group was given 1 of the 3 user stories that they had chosen, and the user stories were split into tasks, and it was estimated how long it would take to complete the user story. 
If we were uncertain of whether we would be able to complete the user story or it came close to the deadline, we were to only assign ourselves to this 1 user story, but if it was clear that another user story could be managed, we were allowed to take another.
but if it was clear that it could be managed another user story we were allowed to take more. 
The product owner wanted to be certain that the user stories, that were picked, were completed and not have unfinished user stories at the end of the sprint.

We were not certain of the time it would take to complete the user story that we had picked, especially as another group had a user story that potentially would conflict with some parts of our user story, hence this conflict had to be resolved.

Our user story was: "As a guardian I would like to confirm with a password that the system is changing to guardian mode so that a citizen cannot gain access to it".

The other group had the user story: "As a guardian, I would like that the app is fully available offline so that I can still use it if the internet is down"

So as our user story was regarding authentication of a guardian entering guardian mode and the others regarding making the entire app work offline, including login authentication, therefore a resolution to who would work on what had to be found, such that we did not get any merge conflicts in the login authentication. 
As our issue contained some frontend and backend while the others only really concerned the backend we agreed that they would resolve the backend work and we would work on the frontend part of our user story and then use the new login functionality that they made.
