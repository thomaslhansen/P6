\section{Crunch}
This crunch period was the first that led up to an actual release of the app.
It was structured such that, the day leading up to the release we were to get as many features in as possible; by having group sessions between pull request authors and reviewers, as described in \autoref{sprint2ImplementationWorkProcess}.

The rest of the crunch was on the release day where the focus was on fixing bugs, as to not introduce more bugs with new features just before releasing. 
We worked on fixing a bug where the logout button did not redirect the user to the login screen after logging out successfully.

\subsection{Implementation}
The original handler for taking a user back to the login screen can be seen in \autoref{snip:BeforeLogoutToLoginScreenBug}, where it simply makes use of a ternary operator to check if the user is logged in or not and depending on the result sends to user to the desired screen.

\begin{lstlisting}[label={snip:BeforeLogoutToLoginScreenBug}, caption={The \_runApp function before the bug fix}, captionpos=b, language=CSharp]
void _runApp() {
  runApp(MaterialApp(
      title: 'Weekplanner',
      home: StreamBuilder<bool>(
          initialData: false,
          stream: di.getDependency<AuthBloc>().loggedIn,
          builder: (_, AsyncSnapshot<bool> snapshot) =>
              // In case logged in show ChooseCitizenScreen, otherwise login
              snapshot.data ? ChooseCitizenScreen() : LoginScreen())));
}
\end{lstlisting}

To implement a fix for the logout button not sending the user back to the login screen, we wrote a function to go home to the initial screen of the app, which was the login screen, this function can be seen in \autoref{snip:GoHomeFunc}.
The goHome function will pop any contexts from the stack, such that it is only the login screen on the stack.

\begin{lstlisting}[label={snip:GoHomeFunc}, caption={The goHome function, which returns to the login screen from anywhere}, captionpos=b, language=CSharp]

  /// Go home is used to pop everything until the navigator is on the
  /// initialRoute.
  static void goHome(BuildContext context) {
    Navigator.of(context).popUntil((Route<dynamic> route) {
      return route.settings.isInitialRoute;
    });
  }
}
\end{lstlisting}

At first glance the difference between \autoref{snip:BeforeLogoutToLoginScreenBug} and \autoref{snip:AfterLogoutToLoginScreenBug} may seem greater than it is.
In reality, the only difference is how the condition is checked, in \autoref{snip:BeforeLogoutToLoginScreenBug} a ternary operator is used and in \autoref{snip:AfterLogoutToLoginScreenBug} an if-else is used.
An if-else is used now instead as the else clause has two statements now instead of one.
The added statement is the goHome function, seen on line $13$ in \autoref{snip:AfterLogoutToLoginScreenBug}.

\begin{lstlisting}[label={snip:AfterLogoutToLoginScreenBug}, caption={The \_runApp function after the bug fix(notice the call to the goHome function)}, captionpos=b, language=CSharp] 
void _runApp() {
  runApp(MaterialApp(
      title: 'Weekplanner',
      home: StreamBuilder<bool>(
          initialData: false,
          stream: di.getDependency<AuthBloc>().loggedIn,
          builder: (BuildContext context, AsyncSnapshot<bool> snapshot) {
            if (snapshot.data) {
              /// In case logged in show ChooseCitizenScreen
              return ChooseCitizenScreen();
            } else {
              /// Not loggedIn pop context to login screen.
              Routes.goHome(context);
              return LoginScreen();
            }
          })));
}
\end{lstlisting}
