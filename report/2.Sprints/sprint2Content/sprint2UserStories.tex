\section{Implementation}

\subsection{Change to guardian mode} \label{changeToGuardianIconImpl}
As first described in \autoref{sprint2planstories} we were to make an icon that would act as a button in the appbar that would allow a guardian to switch the app to guardian mode.
We were also given an icon to use for this task, so we added a button to the appbar in-line with the other currently available buttons, as can be seen in \todo{ref to picture}.

\todo{insert appbar picture}

\begin{lstlisting}[caption={The code used to insert the change to guardian icon in the appbar}, captionpos=b, label=changeToGuardianIconInAppbar]
IconButton(
  icon: Image.asset('assets/icons/changeToGuardian.png'),
  tooltip: 'Skift til værge tilstand',
  onPressed: () {}
)
\end{lstlisting}

The small snippet, seen in \autoref{changeToGuardianIconInAppbar}, is the code used to insert an icon into the appbar.
In the snippet the 'icon' simply refers to the path to the icon that should be used for the button, while the 'tooltip' is a short text that tells what the button is supposed to do.
Lastly the 'onPressed' is there for future use, as it currently does nothing.
The body of it is where any functionality that the button should have is to be written.
The appbar itself consists of more code of course, but this is not relevant to this user story.

\subsection{Confirm password} \label{changeToGuardianDialogBoxImpl}
This user story adds onto the new icon that was inserted into the appbar.
As can be seen when comparing \autoref{changeToGuardianIconInAppbar} and \autoref{changeToGuardianDialogBox} there has only been introduced more to the same 'IconButton' in the 'onPressed' methods body.

\begin{lstlisting}[caption={Dialogbox for confirming password},captionpos=b, label=changeToGuardianDialogBox]
IconButton(
  icon: Image.asset('assets/icons/changeToGuardian.png'),
  tooltip: 'Skift til værge tilstand',
  onPressed: () {
    Alert(
      context: context,
      style: _alertStyle,
      title: 'Skift til værge',
      content: Column(
        children: <Widget>[
          RichText(
            text: TextSpan(
              text: 'Logget ind som ',
              style: DefaultTextStyle.of(context).style,
              children: <TextSpan>[
                TextSpan(
                  text: _authBloc.loggedInUsername,
                  style: const TextStyle(
                    fontWeight: FontWeight.bold
                  )
                ),
              ],
            ),
          ),
          TextField(
            controller: passwordCtrl,
            obscureText: true,
            decoration: InputDecoration(
              icon: const Icon(Icons.lock),
              labelText: 'Adgangskode',
            ),
          ),
        ],
      ),
      buttons: <DialogButton>[
        DialogButton(
          onPressed: () {
            login(context, _authBloc.loggedInUsername,
                passwordCtrl.value.text);
            Routes.pop(context);
          },
          child: const Text(
            'Bekræft',
            style: TextStyle(color: Colors.white, fontSize: 20),
          ),
          color: const Color.fromRGBO(255, 157, 0, 100),
        )
      ]
    ).show();
  },
)
\end{lstlisting}

%attempted rewrite of commented paragraph below
The implementation assumes that the guardian that first logged in is also the one that will change to guardian mode, so that when attempting to change to guardian mode only the password is needed.
Even though the real-world usage of this functionality might deviate from the assumption, this is how the PO-group specified the issue.
This was achieved through saving the username used when first logging in, and simply retrieve and display it when needed.
This can be seen in the parameters given to the 'RichText' on line $11$ in \autoref{changeToGuardianDialogBox}, where one of the children to this is a 'TextSpan' where we use the username that was used at login and ensure that the username is bold to emphasize it.

The password is retrieved by using the 'TextField' on line $25$ in \autoref{changeToGuardianDialogBox} and a 'controller' parameter that is used to pass on the password to the login function, which is called when the user presses the confirm button.

The confirm button is seen on line $35$ in \autoref{changeToGuardianDialogBox}, this is a parameter to the dialogbox.
When pressed it will call a 'login' function with the username that was displayed and the password that was entered into the 'TextField' and then close the dialogbox.
The login function will in the end make a call to the API to authenticate the user and set their state to logged in.

The final dialogbox that the user will see can be seen in \autoref{changeToGuardianDialogBoxImage} and it is structered such that the title is 'Change to guardian', a text that says what username you are about to use 'logged in as Graatand', a text input field that is labeled with 'Password' and lastly a 'Confirm' button.

%The 'onPressed' function will open a new dialog box that will be centered in the screen and grey out anything in the background, such that the dialogbox is now in focus and presented as such cleary to the user.
%In this dialogbox the user will see the username of the guardian that was used at the time of logging in.
%So in the case of the example in \autoref{changeToGuardianDialogBoxImage} the username used to log in was 'Graatand' and therefore the textfield prompting for a password, below the shown username, needs to get the matching password for that username.
%Lastly there is a confirm button to attempt to change to guardian mode.
%All of this can be seen in \autoref{changeToGuardianDialogBoxImage}.

\figur{0.7}{images/changeToGuardianDialogbox.png}{Dialogbox for changing to guardian mode}{changeToGuardianDialogBoxImage}

This implementation is clearly not the same as what was specified in the design guide from \autoref{sprint2planstories}, but when we made it according to the design guide at first the PO group stated that they would rather have it as a dialog box.
So instead of the implementation being a screen, now it is a dialog box that covers the center of the screen and greys out anything in the background.

The implementation also had an issue associated with it, as when the user would enter an incorrect password they would be logged out of the app, as their credentials no longer would be valid.
This was an issue with how the logged in status of the app was handled and is therefore seperate of this user story.
For this reason we simply reported it to the PO group and made a bug report on the GitHub repository.
Also at the time of developing this implementation the idea of guardian mode and citizen mode was yet to be implemented, which is the main reason why when switching to guardian mode the app simply makes a login attempt.

%don't know if this should technically be here or somewhere else or perhaps be incorporated into the other subsections here.
\subsection{Process}
Our workflow during the development of the user stories were very straight forward, we started by adding the icon as described in \autoref{changeToGuardianIconImpl}.
Once we were done we were initially supposed to create a pull request on GitHub, but this seemed like it would just cause unnecessary wait time for it to be reviewed and then merged with the app.
Therefore instead we continued working immediately on the implementation from \autoref{changeToGuardianDialogBoxImpl}.

Only when we were done with both user stories did we create a pull request and awaited review.
It was here that the PO group asked if we could make it a dialog box instead of a screen.

So we made the button open a dialogbox instead of a screen and created it as it is seen in \autoref{changeToGuardianDialogBoxImpl}.
Once we had a working dialogbox with the bare necessities, we were nearing the end of the sprint and it was decided that any groups that had open pull requests should have at least one of the reviewers of that pull request sitting with them to get 'live' feedback.

So with the reviewer present we were having a group programming session with them and were able to discuss any adjustments such as whether or not to have the close icon on dialogbox, as it would close if you pressed outside of it.
This way of doing a last crunch was very effective and we got our user story done quickly.
