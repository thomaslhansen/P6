\section{User stories implementation}

\subsection{Switch to guardian mode}
As first described in \autoref{sprint2planstories} we were to make an icon that would act as a button in the appbar that would allow a guardian to switch the app to guardian mode.
We were also given an icon to use for this task, so we added a button to the appbar in-line with the other currently available buttons.

\todo{insert appbar picture}

% æ, ø og å encodes rigtigt i listings
\lstset{
  literate=%
  {æ}{{\ae}}1
  {å}{{\aa}}1
  {ø}{{\o}}1
  {Æ}{{\AE}}1
  {Å}{{\AA}}1
  {Ø}{{\O}}1,
  extendedchars=\true,
  inputencoding=ansinew,
  language=CSharp
}

\lstset{
  caption=The code used to insert the change to guardian icon in the appbar,
  label=changeToGuardianIconInAppbar
}

\begin{lstlisting}
IconButton(
  icon: Image.asset('assets/icons/changeToGuardian.png'),
  tooltip: 'Skift til værge tilstand', 
  onPressed: () {}
)
\end{lstlisting}

The small snippet, seen in \autoref{changeToGuardianIconInAppbar}, is the bare minimum needed to insert an icon into the appbar. 
In the snippet the 'icon' simply refers to the path to the icon that should be used for the button, while the 'tooltip' is a short text that tells what the button is supposed to do. 
Lastly the 'onPressed' is there for future use, as it currently does nothing. 
The body of it is where any function that the button should have should be written.
The appbar itself consists of more code of course, but this is not relevant to this user story.

\subsection{Confirm password}
This user story adds onto the new icon that was inserted into the appbar. 
As can be seen in \autoref{changeToGuardianDialogBox} the addition occurred in the body of the 'onPressed: () {}' function.

\lstset{
  caption=Dialog box for confirming password,
  label=changeToGuardianDialogBox
}

\begin{lstlisting}
IconButton(
  icon: Image.asset('assets/icons/changeToGuardian.png'),
  tooltip: 'Skift til værge tilstand',
  onPressed: () {
    Alert(
      context: context,
      style: _alertStyle,
      title: 'Skift til værge',
      content: Column(
        children: <Widget>[
          RichText(
            text: TextSpan(
              text: 'Logget ind som ',
              style: DefaultTextStyle.of(context).style,
              children: <TextSpan>[
                TextSpan(
                  text: _authBloc.loggedInUsername,
                  style: const TextStyle(
                    fontWeight: FontWeight.bold
                  )
                ),
              ],
            ),
          ),
          TextField(
            controller: passwordCtrl,
            obscureText: true,
            decoration: InputDecoration(
              icon: const Icon(Icons.lock),
              labelText: 'Adgangskode',
            ),
          ),
        ],
      ),
      buttons: <DialogButton>[
        DialogButton(
          onPressed: () {
            login(context, _authBloc.loggedInUsername,
                passwordCtrl.value.text);
            Routes.pop(context);
          },
          child: const Text(
            'Bekræft',
            style: TextStyle(color: Colors.white, fontSize: 20),
          ),
          color: const Color.fromRGBO(255, 157, 0, 100),
        )
      ]).show();
  },
)
\end{lstlisting}

The 'onPressed' function will open a new dialog box that will be centered in the screen and grey out anything in the background, such that the dialogbox is now in focus and presented as such cleary to the user. 
In this dialogbox the user will see the username of the guardian that was used at the time of logging in. 
So in the case of the example in \autoref{changeToGuardianDialogBoxImage} the username used to log in was 'Graatand' and therefore the textfield prompting for a password, below the shown username, needs to get the matching password for that username. 
Lastly there is a confirm button to attempt to change to guardian mode.
All of this can be seen in \autoref{changeToGuardianDialogBoxImage}.

\figur{0.7}{images/changeToGuardianDialogbox.png}{Dialogbox for changing to guardian mode}{changeToGuardianDialogBoxImage}

This implementation is not the same as what was specified in the design guide, but as we made it according to the design guide the PO group stated that they would rather have it as a dialog box. 
So instead of the implementation being a screen, now it is a dialog box that covers the center of the screen and greys out anything in the background.
The implementation also had an issue associated with it, as when the user would enter an incorrect password they would be logged out of the app, as their credentials no longer would be valid. 
This was an issue with how the logged in status of the app was handled and is therefore seperate of this user story. 
For this reason we simply reported it to the PO group and made a bug report on the GitHub repository.
