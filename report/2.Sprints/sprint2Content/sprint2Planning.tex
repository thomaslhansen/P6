\section{Planning}
For this sprint the planning was held in a more relaxed fashion.
It was more of an introduction to the goals of the sprint, rather than an actual planning meeting.
The reasoning is that we believed, from the feedback of sprint 1's retrospective, that we would achieve a faster and easier workflow by using the issue system from GitHub to see what current user stories were available and a priority for the sprint. 

This switch in workflow also means that we are no longer using scrum of scrums, at least not in a traditional sense, but we are using a lot of the elements that it provides.
The only "real" change is the way that we "plan" a sprint; now we use GitHub issues for user stories, so each issue on GitHub is equivalent to a user story and we also do not assign groups to user stories at the planning meeting. 
Instead we will be assigning ourselves to a fitting amount of user stories, that we choose in collaboration with the PO group, at any time during the sprint, so if we run out of user stories we pick a new one from GitHub. 

The sprint was to start on the 18\textsuperscript{th} of March and end on the 5\textsuperscript{th} of April. This gave us 15 workdays for this sprint.

\subsection{Goals}
The goal of this sprint was to make the flutter rewrite of the app more user friendly and improve the user experience of the app in comparison to the, now, old Xamarin implementation.
In order to make the app more user friendly we were to implement more basic functionality into the flutter version of the app, as it was very lacking at this point.
Most of the basic functionality was already described as user stories and available on GitHub. 
These user stories were mainly related to making a flutter version of the old Xamarin features and in the process improve and update them in accordance with the goals.
To improve the user interface of the app we were to either use the prototypes that they had already attached to the user story or follow the design guide that the PO group made.

\subsection{Our initial user stories} \label{sprint2planstories}
After the introduction to the sprint, the PO group went out to each group to get some perspective of how far along everyone were with their current user stories, that remained from sprint 1, and to check whether or not a group was free to take new user stories.

We were essentially done with our user story from the previous sprint, and only waiting for another groups changes to the app to be merged with our changes. 
As their task was to update the appbar at the top and we were to add functionality to the logout button in that appbar.
So we picked some high priority user stories in collaboration with the PO group.
These user stories were:

%What
\begin{enumerate}
\item "As a guardian I would like to be prompted for a password when changing from citizen to guardian mode, so that a citizen cannot access the guardian mode"
\item "As a guardian I would like to be able to switch from citizen mode to guardian mode with the press of a button so that I can change the users schedule without restarting the application"
\end{enumerate}

The first user story depends on the second user story, when the app is in citizen mode a guardian can with their credentials log in as a guardian, which is in essence an administrator, by using the button that the second user story is about. 

%Why
We chose these user stories as our tasks for this sprint to give the ability for a guardian to switch back into guardian mode, as currently the only way to do so was to logout and then login again. 
So we wanted to make it more convenient for the guardians to switch to guardian mode and make any necessary changes, while also making sure that a citizen does not get access to guardian mode.

%Expected result
Thanks to the PO group we had a clear goal for our user stories and how they would end up looking in the app. 
This was thanks to the design guide that had been made between the customers and the PO group.

We expected the result of the 2\textsuperscript{nd} user story to be the addition of the icon to the appbar and by clicking the icon the user would be switched to guardian mode. 
The icon of the button to switch to guardian mode can be seen in \autoref{changeToGuardianIcon}.
However, as we also took our 1\textsuperscript{st} user story into account immediately, so we wanted it to ask for password confirmation. 
So instead of just switching to guardian mode it would now go to the password confirmation screen, seen in \autoref{confirmPassScreenProto}, and only switch once a correct password was given.

\figur{0.2}{images/changeToGuardian.pdf}{Icon for changing to guardian mode}{changeToGuardianIcon}
\figur{1}{images/citizenToGuardianPassScreen.png}{Prototype for confirming password}{confirmPassScreenProto}

%How long will it take
We do not expect to use a lot of time on these user stories, but we do expect them to take around a work week, as we are only getting started with the flutter workprocess.
