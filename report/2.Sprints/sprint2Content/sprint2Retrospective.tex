\section{Retrospective}
To propose ideas and suggestions for improving the workflow of the Giraf project, we used \href{https://dotstorming.com/}{dotstorming} to create a brainstorm of all the ideas. 
Once we had a board full of ideas and suggestions everyone were able to pick their top three ideas to vote for.

The changes and additions to the Giraf process are listed below.

\begin{itemize}
  \item Make the crunch a day where we sit and work together in one large room
  \item When making a pull request, you should always include a screenshot of what you made
  \item Linter should give an error instead of a warning for not documenting public members
  \item Have a testing workshop event
  \item Make the PO group approve every Weekplanner pull request to check the design
  \item As a reviewer of a pull request it is now encouraged to sit with the author of it to speed it up
  \item Only the author of a pull request should commit, merge, and delete their branch.
  \item Meetings should always be in the calendar, with time, location, and description
  \item A review checklist should now always be in added to a pull request (See \autoref{app:code-review-checklist})
\end{itemize}

We agreed with these new changes to the process.
The changes that we particularly agreed with were the Linter change, the author being in control of their pull request, and the meetings being in the calendar.
Changing the Linter warning to be an error will not only help us, but also anyone trying to read the code in the future.
Making the author of a pull request the one who controls it, was also welcome as it is the standard practice in most projects.
Having all meetings in the calendar was a nice change from it sometimes being there and sometimes only being on Slack.
So overall we are satisfied with the outcome of the retrospective.
