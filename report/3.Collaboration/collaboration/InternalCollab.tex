During the project period the group has been collaborating both across the entire project and within the group itself.
The work done during the sprint, as described in \autoref{PART:Sprints}, will be the base for this chapter.
In the following chapter the different aspects of this collaboration will be elaborated upon and evaluated.

\section{Process}
Throughout the project all work has been divided into two categories: product and report.
Within both categories a development process model based on the Waterfall model has been used to conduct the work.
In the following the different categories will be elaborated and explained.

\subsection{Product}
The product development has throughout the different sprints all been planned during the planning phase of each sprint.
After each planning phase, the group were assigned to a number of issues by the PO group as described in \autoref{PART:Sprints}.
Once each sprint was planned the group internally assigned the issues to the group members best suited to solve the issue.

During the sprint it would be the responsibility of the assigned member to make sure that the issue would be solved at the end of the sprint.
This meant that it was not necessarily the assigned member who where to solve the issue, but the assigned member could get help from the other group members.

\subsection{Report}\label{SEC:InternalCollabReport}
Throughout the project period the group's main focus has been on the product.
This meant that while the development of the product was still going on all focus was based on that.
However, when time allowed it, the rest of the group would work on the report.

The report was managed via GitHub and written in \LaTeX, which allowed the group members to each write on the report at the same time, and when a member had finished a section, it would be merged together into the final version.
Before written work could be merged it had to go through a review process, just like when working with code.
This process ensured that the written work was of an acceptable quality, as well as forcing all members to be familiarized with more text than they wrote themselves.

\section{Teamwork}
Throughout the project the group has been working together, for some group members for the first time, in various ways.
The following section will elaborate on the teamwork in different aspects.

\subsection{Communication}
Throughout the project the primary mean of communication has been via Slack.
By using Slack it meant that each group member was contactable most of the normal working hours.
This made it easy to communicate throughout the group.
When a given group member was unreachable on Slack, the phone network was utilized as a secondary mean of communication.
It was agreed upon that Slack was not the ideal communication tool.
However, it is the best solution, that is known and available to us.

When communicating with the staff at AAU the primary mean of communication was via email.
If, however, an unexpected long response time was experienced, it was found to be more effective to make a physical appearance in the staff member's office.
This was especially useful when dealing with ITS.

\subsection{Attendance}
During the start up phase of the project there was a desire within the group to have regular meetings everyday from 0815-1600 o'clock.
However, during the project period these times where largely abandoned, but the common agreement was still that the group would sign in between 0800 and 0900 o'clock.
It was deemed unnecessary to dictate a specific time of meeting, however a single timeslot made it easier for the group.
The finish time was also abandoned since there was no reason to stay at the group room if the productivity had dropped.
Hence there was no formal specification on when and where the group work should be conducted, but there was an oral agreement on how the work should be conducted.
