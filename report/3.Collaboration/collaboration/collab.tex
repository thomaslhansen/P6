\section{API Update}\label{CollabAPIUpdate}
The following section describes a collaboration between group SW602F19, SW609F19, SW611F19, and SW612F19, and as such, will appear in all of the group's reports as is written here.

With both a new ability for the frontend to search for Pictograms and a monitoring tool for the servers, we noticed transfers of several gigabytes between the database and the backend every time the frontend searched. 
Performing several searches in a short timespan was guaranteed to crash the backend or the database.

A more in depth inspection found that the database stored all the pictogram images, which requires the database to extract vast amounts of binary-image-data for each pictogram related query. 
Furthermore, the API ensured that all pictograms loaded should be Base64 encoded on load, which meant that all pictograms traveling through the API, had to go through relatively large processing.

We decided to extract the pictograms from the database and instead store them as files on disk, which relieved the database of the large transfers. 
The extraction itself was not enough to stop the crashing of the API, as the pictograms were still being Base64 encoded. 
We decided to remove the image property from the pictogram DTO, as it contained the Base64 encoded image, thereby halting all encoding.

The solution showed a massive improvement in both the amount of data transferred and response time as seen on \autoref{tab:newApi}.

\begin{table}[H]
    \centering
    \begin{tabular}{|l|r|r|}
    \hline
                    & Data transferred  & Response time \\ \hline
        Old         & 14020 MB         & 1482 ms      \\ \hline 
        New         & 60 MB             & 30 ms         \\ \hline
        Improvement & 23400\%          & 4900\%       \\ \hline
    \end{tabular}
    \caption{API timings before and after changes}
    \label{tab:newApi}
\end{table}

The most significant improvement was extracting the pictogram images from the database onto a network file share, which relieved the database immensely. 
This improvement has other advantages than just being faster; it also puts less load on the database, allowing the database to handle more simultaneous connections and it massively reduced the size of the database, making it much easier to search in.

After the improvements, searching for pictograms takes a fraction of the time it took before the improvement. 
The specific improvement also translates to an overall improvement of the system, as the system can allocate more resources to other parts of the API.
