\section{Backend Workshop} \label{SEC:workshopAPIcollab}
The following section describes a collaboration between all the groups of the Giraf project of 2019. 
On the 29th of April, the backend meta group held a workshop, where the agenda was doing some collaborative fixing of unit tests, identifying issues, and fixing these issues. 
The overall goal of the workshop was to share knowledge about backend development since the level of competence with e.g. WEB-API development varied a lot between the backend meta group members.

Before the meta group workshop, several unit tests failed due to the performance enhancements performed by the server meta group (see \autoref{CollabAPIUpdate}). 
We set out to fix these unit tests in a mob programming manner so as to include all members of the meta groups in resolving these issues.
The mob programming was also intended to be an activity in which less experienced members could become more comfortable with the WEB-API repository and the unit testing of endpoints.

All of the unit tests were updated to correctly reflect how the code behaves today.
As some of the tests for the PictogramController failed after the new changes were made to how pictograms were stored (one of the optimizations made by the server meta group), the pictogram tests also had to be changed to create test images that could be used in every test.
This was achieved by ensuring that each test would generate all test images.
As each test was now independent, they could no longer fail if a different test had modified the test image.

Additionally, it was discovered that some of the existing unit tests were of lesser quality, which should probably be addressed in the future if code quality of the backend is of priority.

By lesser quality we mean, primarily, the improper use of correct assertions. 
Also the setup of the test was not following any known standards, e.g. the triple A of arrange, act, assert. 
This resulted in some tests being unable to stand alone.

\begin{lstlisting}[label={snip:OldTestBeforeUpdate}, caption={The old test after images are served from disk}, captionpos=b, language=CSharp] 
[Fact]
public void ReadRawPictogramImage_GetPublic_Success()
{
try
{
    var pc = initializeTest();
    testContext.MockUserManager.MockLoginAsUser(_testContext.MockUsers[ADMIN_DEP_ONE]);
    var res = pc.ReadRawPictogramImage(DEP_ONE_PROTECTED_PICTOGRAM);

    Assert.True(res.IsCompleted);
    Assert.IsType<FileContentResult>(res.Result);

    var fileContent = ((FileContentResult)res.Result);

    // Get the expected image an convert to eight bit int array so we can compare with actual returned image
    Assert.Equal("image/png", fileContent.ContentType);
    var testImageTo8BitIntArray = Convert.FromBase64String(System.Text.Encoding.UTF8.GetString(_testContext
    .MockPictograms
    .FirstOrDefault(mp => mp.Id == DEP_ONE_PROTECTED_PICTOGRAM)?.Image));

    // Check that we read the correct image
    Assert.Equal(testImageTo8BitIntArray, fileContent.FileContents);
}
catch(Exception)
{
    Assert.True(false);
}
\end{lstlisting}

In \autoref{snip:OldTestBeforeUpdate} and \autoref{snip:NewUpdatedTest}, both the old and the new unit tests are seen. 
The two tests do assert the same issue, however the new one no longer uses pictograms stored inside the database.

\begin{lstlisting}[label={snip:NewUpdatedTest}, caption={The updated test after images are served from disk}, captionpos=b, language=CSharp] 
[Fact]
public void ReadRawPictogramImage_GetPublic_Success()
{
    var pc = initializeTest();
    testContext.MockUserManager.MockLoginAsUser(_testContext.MockUsers[ADMIN_DEP_ONE]);
    var res = pc.ReadRawPictogramImage(DEP_ONE_PROTECTED_PICTOGRAM);

    Assert.True(res.IsCompleted);
    Assert.IsType<PhysicalFileResult>(res.Result);

    var fileContent = ((PhysicalFileResult)res.Result);

    // Get the expected image an convert to eight bit int array so we can compare with actual returned image
    Assert.Equal("image/png", fileContent.ContentType);
            
    var expectedImage = _loadPictogramFromDisk(DEP_ONE_PROTECTED_PICTOGRAM);

    // Check that we read the correct image
    Assert.Equal(expectedImage, File.ReadAllBytes(fileContent.FileName));      
}
\end{lstlisting}

The issues \href{https://github.com/aau-giraf/web-api/issues/18}{\texttt{\#}18}, \href{https://github.com/aau-giraf/web-api/issues/17}{\texttt{\#17}} and \href{https://github.com/aau-giraf/web-api/issues/15}{\texttt{\#}15} were identified and initiated during the workshop.
Issue \href{https://github.com/aau-giraf/web-api/issues/15}{\texttt{\#}15} revolved around the warnings that were thrown when building the WEB-API.
Both issue \href{https://github.com/aau-giraf/web-api/issues/18}{\texttt{\#}18} and \href{https://github.com/aau-giraf/web-api/issues/17}{\texttt{\#}17} revolved around adding activity endpoints.
These endpoints were found necessary since currently adding, deleting or updating an activity required using the update week endpoint, updating an entire week plan instead of a single activity. 
 
In conclusion, we think the workshop served as a good introduction to development in the WEB-API repository, and would advice future groups to do the same.
In retrospect we could have benefited from having the workshop sooner, but the Weekplanner was of higher priority in the initial sprints and the WEB-API was already in a functioning condition.

