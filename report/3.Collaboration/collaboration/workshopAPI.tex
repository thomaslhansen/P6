\section{Backend Workshop} \label{SEC:workshopAPIcollab}
The following section describes a collaboration between all the groups of the Giraf project of 2019. 
On the 29th of April, the backend skill group\footnote{This is equivalent to meta groups, but the other groups called it skill groups} held a workshop, where the agenda was doing some collaborative fixing of unit tests, identifying issues, and fixing these issues. 
The overall goal of the workshop was to share knowledge about backend development since the level of competence with e.g. REST API development varied a lot between the backend skill group members.

Before the skill group workshop, several unit tests failed due to the performance enhancements performed by the server skill group (see \autoref{CollabAPIUpdate}). 
We set out to fix these unit tests in a mob programming manner so to include all members of the skill groups in resolving these issues.
The mob programming was also intended to be an activity in which less experienced members could become more comfortable with the REST API repository and the unit testing of endpoints.

All of the unit tests were updated to correctly reflect how the code behaves today.
As some of the tests for the PictogramController failed after the new changes were made to how pictograms were stored (one of the optimizations made by the server skill group), the pictogram tests also had to be changed to create test images that could be used in every test.
This was achieved by ensuring that each test would generate all test images.
As each test was now independent, they could no longer fail if a different test had modified the test image.

Additionally, it was discovered that some of the existing unit tests were of lesser quality, which should probably be addressed in the future if code quality of the backend is of priority.
\todo{Note from collab document: Should perhaps be addressed as issue or in the report}

The issues \href{https://github.com/aau-giraf/web-api/issues/18}{\texttt{\#}18}, \href{https://github.com/aau-giraf/web-api/issues/17}{\texttt{\#17}} and \href{https://github.com/aau-giraf/web-api/issues/15}{\texttt{\#}15} were identified and initiated during the workshop.
Issue \href{https://github.com/aau-giraf/web-api/issues/15}{\texttt{\#}15} revolved around the warnings that were thrown when building the REST API.
Both issue \href{https://github.com/aau-giraf/web-api/issues/18}{\texttt{\#}18} and \href{https://github.com/aau-giraf/web-api/issues/17}{\texttt{\#}17} revolved around adding activity endpoints.
These endpoints were found necessary since currently adding, deleting or updating an activity required using the update week endpoint, updating an entire week plan instead of a single activity. 
 
In conclusion, we think the workshop served as a good introduction to development in the REST API repository, and would advice future groups to do the same.
In retrospect we could have benefited from having the workshop sooner, but the weekplanner was of higher priority in the initial sprints and the REST API was already in a functioning condition.

